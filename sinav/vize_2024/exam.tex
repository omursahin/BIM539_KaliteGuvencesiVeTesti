\documentclass{exam}
\cfoot[]{Sayfa \thepage}
\usepackage{courier} %% Sets font for listing as Courier.
\usepackage{listings, xcolor}
\lstset{
	tabsize = 4, %% set tab space width
	showstringspaces = false, %% prevent space marking in strings, string is defined as the text that is generally printed directly to the console
	numbers = left, %% display line numbers on the left
	commentstyle = \color{black}, %% set comment color
	keywordstyle = \color{black}, %% set keyword color
	stringstyle = \color{red}, %% set string color
	rulecolor = \color{black}, %% set frame color to avoid being affected by text color
	basicstyle = \small \ttfamily , %% set listing font and size
	breaklines = true, %% enable line breaking
	numberstyle = \tiny,
}
\begin{document}
	\addpoints
	\settabletotalpoints{100}
	\hqword{Sorular}
	\htword{Toplam}
	\hpword{Puan}
	\hsword{Alınan}
	\pointname{ Puan}
	
	\begin{center}
		
		\fbox{\fbox{\parbox{5.5in}{\centering
					BS450 Yazılım Test Mühendisliği
					
					\textbf{Süre}: 45 Dakika}}}
	\end{center}
	
	\begin{center}
		\gradetable[h][questions]
	\end{center}
	
	\vspace{5mm}
	\makebox[0.75\textwidth]{Ad Soyad:\enspace\hrulefill}
	
	\vspace{5mm}
	\makebox[0.75\textwidth]{Öğrenci Numarası:\enspace\hrulefill}\\\\
	
	
	
	\begin{center}
		\textbf{SORULAR}
	\end{center}
	\begin{questions}

		\question[25] Exhaustive testin zorluklarını açıklayınız.
	
		\question[25] RIPR modeli nedir? Açıklayınız.
		
		\question[25] Test üçgenindeki katmanlar nelerdir? Her bir katmanı açıklayınız.
		\question Aşağıdaki kod parçacığı bir dizide bulunan sıfırlardan en son sıradakinin indisini döndürmektedir. Bu kod parçacığını dikkate alarak;
		\begin{lstlisting}[language = Java , frame = trBL , firstnumber = last , escapeinside={(*@}{@*)}]
public static int lastZero(int[] x) {
    for (int i = 0; i < x.length; i++) {
		if (x[i] == 0) {
		  return i;
		}
	}
	return -1;
}
		\end{lstlisting}
		\begin{parts}
			\part[10] Koddaki kusur nedir ve hangi satırda bulunmaktadır? 
			\part[15] Kusuru çalıştıracak bir test yazınız.
		\end{parts}
		\hrulefill
		\begin{center}
			\textbf{CEVAPLAR}
		\end{center}

		
	\end{questions}
	
\end{document}